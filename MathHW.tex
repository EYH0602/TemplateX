\documentclass[12pt]{article}

\usepackage[english]{babel}
\usepackage[utf8]{inputenc}
\usepackage{amsthm}
\usepackage{amsmath,amssymb}
\usepackage{parskip}
\usepackage{graphicx}
\usepackage[normalem]{ulem}

% Margins
\usepackage[top=2.5cm, left=3cm, right=3cm, bottom=4.0cm]{geometry}
% Colour table cells
\usepackage[table]{xcolor}

% Get larger line spacing in table
\newcommand{\tablespace}{\\[1.25mm]}
\newcommand\Tstrut{\rule{0pt}{2.6ex}}         % = `top' strut
\newcommand\tstrut{\rule{0pt}{2.0ex}}         % = `top' strut
\newcommand\Bstrut{\rule[-0.9ex]{0pt}{0pt}}   % = `bottom' strut

%%%%Changes paragraph indentation to 0.5in
\setlength{\parindent}{0.25in}
%%%%Begin body of paper here

\begin{document}

\begin{center}
    \textbf{TITLE}
\end{center}

NAME: 

1. Decide which of the following represent true statements about the nature of sets.
For any that are false, provide a specific example where the statement in question does not hold.

(a) If $A_1 \supseteq A_2 \supseteq A_3 \supseteq A_4 \cdots$ are all sets containing an infinite number of elements,
then the intersection $\bigcap\limits_{n=1}^{\infty} A_n$ is infinite as well.

False.
\begin{proof}
    \begin{align*}
        \text{Let } A_1              & = \{1,2,3,\ldots\}                             \\
        A_2                          & = \{2,3,4,\ldots\}                             \\
        \ldots                                                                        \\
        \text{for number } m, m > n, & \text{ then } m \in \bigcap_{n=1}^{\infty} A_n \\
        \text{and } m < m+1          & \implies m \not\in A_{m+1}                     \\
        \implies                     & \bigcap_{n=1}^{\infty}A_n = \emptyset
    \end{align*}
\end{proof}

(b) If $A_1 \supseteq A_2 \supseteq A_3 \supseteq A_4 \cdots$ are all finite, nonempty sets of real numbers,
then the intersection $\bigcap\limits_{n=1}^{\infty} A_n$ is finite and nonempty.

(c) $A \cap (B \cup C) = (A \cap B) \cup C$
\begin{proof}
    \begin{align*}
        \text{Let set } A & = \{1,2\}                        \\
        B                 & = \{2,3\}                        \\
        C                 & = \{3,4\}                        \\
        \text{then, }                                        \\
        A \cap (B \cup C) & = \{1,2\} \cap \{2,3,4\} = \{2\} \\
        (A \cap B) \cup C & = \{2\} \cup \{3,4\} = \{2,3,4\}
    \end{align*}
\end{proof}


\end{document}
