\documentclass[12pt]{article}

\usepackage[english]{babel}
\usepackage[utf8]{inputenc}
\usepackage{amsthm}
\usepackage{amsmath,amssymb}
\usepackage{parskip}
\usepackage{graphicx}
\usepackage[normalem]{ulem}

% theorem styles

\newtheorem{corollary}{\bf Corollary}
\newtheorem{definition}{\bf Definition}
\newtheorem{proposition}{\bf Proposition}
\newtheorem{Axiom}{\bf Axiom}
\newtheorem{theorem}{\bf Theorem}
\newtheorem{lemma}{\bf Lemma}
\newtheorem{example}{\bf Example}
\newtheorem{remark}{\bf Remark}
\newtheorem{Algorithm}{\bf Algorithm}[section]
\newcommand{\reff}[1]{Figure~\ref{#1}}
\newcommand{\reftb}[1]{Table~\ref{#1}}

\newcommand\round[1]{\left[#1\right]}

% mathbb and mathcal often used letters
\newcommand{\Z}{\mathbb{Z}}
\newcommand{\N}{\mathbb{N}}
\newcommand{\Q}{\mathbb{Q}}
\newcommand{\R}{\mathbb{R}}
\newcommand{\C}{\mathbb{C}}
\newcommand{\I}{\mathbb{I}}

% text command
\newcommand{\suchthat}{\text{ s.t. }}
\newcommand{\andtext}{\text{ and }}
\newcommand{\ortext}{\text{ or }}


\usepackage{mathtools}

\DeclarePairedDelimiter\abs{\lvert}{\rvert}%
\DeclarePairedDelimiter\norm{\lVert}{\rVert}%

% Swap the definition of \abs* and \norm*, so that \abs
% and \norm resizes the size of the brackets, and the 
% starred version does not.
\makeatletter
\let\oldabs\abs
\def\abs{\@ifstar{\oldabs}{\oldabs*}}
%
\let\oldnorm\norm
\def\norm{\@ifstar{\oldnorm}{\oldnorm*}}
\makeatother

% double indent
\newcommand{\iindent}{\indent\indent}

% Margins
\usepackage[top=2.5cm, left=3cm, right=3cm, bottom=4.0cm]{geometry}
% Colour table cells
\usepackage[table]{xcolor}

% Get larger line spacing in table
\newcommand{\tablespace}{\\[1.25mm]}
\newcommand\Tstrut{\rule{0pt}{2.6ex}}         % = `top' strut
\newcommand\tstrut{\rule{0pt}{2.0ex}}         % = `top' strut
\newcommand\Bstrut{\rule[-0.9ex]{0pt}{0pt}}   % = `bottom' strut

%%%%Changes paragraph indentation to 0.5in
\setlength{\parindent}{0.25in}
%%%%Begin body of paper here

\begin{document}


\begin{center}
    \textbf{TITLE}
\end{center}

\textbf{NAME:} 

\section*{Example}

1. Let $y_1=6$, and for each $n\in \N$ define $y_{n+1}=\frac{2y_n-6}{3}$.

(a) Use induction to prove that the sequence satisfies $y_n>-6$ for all $n\in \N$.

\begin{proof}
    We use induction on $n$.

    \textit{Base Case}: $n=1$
    \begin{align*}
        y_{n+1} = y_2 = \frac{2y_1-6}{3} = 2 > -6
    \end{align*}
    \indent We prove the base case.

    \textit{Inductive Step}: Let $k \in \N, k \geq n$, suppose $y_k > -6$,
    \begin{align*}
        y_{k+1} & = \frac{2y_{k}-6}{3} \\
        & > \frac{2(-6) - 6}{3} \\
        & > \frac{-18}{3} \\
        & > -6
    \end{align*}
    \indent This proves the inductive step.

    Conclude by Induction, the result holds.
\end{proof}

(b) Use another induction argument to show the sequence $(y_1,y_2,y_3,...)$ is decreasing.

\begin{proof}
    We use strong induction on $n$.

    
    \textit{Base Case}: $n=1$ and $n=2$
    \begin{align*}
        n=1: y_1 = 6 \\
        n=2: y_2 = 2 < y_1
    \end{align*}
    \indent the base case holds.

    \textit{Induction Step}: Let $k \in \N, k \geq 2$ and suppose $y_n < y_{n-1} \forall n \leq k$, then we have
    \begin{align*}
        y_{k+1} & = \frac{2y_k-6}{3}, y_k = \frac{2y_{k-1}-6}{3} \\
        \implies y_{k+1} - y_k & = \frac{(2y_k-6) - (2y_{k-1}-6)}{3} \\
        y_{k} < y(k-1) & \implies (2y_k-6) < (2y_{k-1}-6) \\
        & \implies y_{k+1} < y_k
    \end{align*}
    \indent The inductive step holds. Prove by strong induction, the statement holds. \\
\end{proof}


\end{document}
